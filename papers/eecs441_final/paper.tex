\documentclass[12pt]{article}
\usepackage[margin=1in]{geometry}
\usepackage{graphicx}
\usepackage{setspace}
\usepackage{url}

\newenvironment{tscode}
{\begin{list}{}{\setlength{\leftmargin}{1em}}\item\scriptsize\bfseries}
{\end{list}}

\title{Tecellate: A Distributed Environment for Ad Hoc Wireless Network Simulations}
\author{
        Steve Johnson (srj15@case.edu)\\
        Case Western Reserve University\\
        \and Tim Henderson (tah35@case.edu)\\
        Case Western Reserve University
}
\date{\today}

\begin{document}
\doublespacing
\maketitle

% \begin{multicols}{2}

\begin{abstract}
    Tecellate is a simulation in which autonomous agents can move in a grid and communicate using
    semi-realistic radio signals. The simulation itself can be distributed over multiple machines to
    allow for very large simulations. The system allows researchers to test various wireless network
    protocols and algorithms under arbitrarily harsh conditions.
    
    The usefulness of the simulation was tested using an approximated IP stack. Our initial results
    show that the simulation produces useful information, though more work is required to improve
    its accuracy and suitability for research.
\end{abstract}

\section{Introduction}

Computer network modeling and simulation has become an important tool in the arsenal of any computer
networks researcher. It has only become more important as increasingly research into mobile
and autonomous devices matures. Large scale simulations of thousands to hundreds of thousands of
devices become increasingly crucial to validate techniques without having to spend thousands of
dollars on costly equipment. To support such simulations we present Tecellate, a horizontally
scalable distributed network simulator.

Tecellate simulates the physical layer of the network. All \textbf{agents} (actors in the
simulation) have a ``radio'' to send and receive byte strings. To approximate the real world, these
byte strings may be probabilistically corrupted as a function of distance and interference. To
provide reliable communication between arbitrary agents, users must implement the link, IP, and
transport layers. As a resource to users wishing to model the traditional TCP/IP stack the authors
provide a partial implementation up to the IP layer.\footnote{
  We hope to expand this implementation to include the TCP and UDP transport layers as well.
}

In the physical world, devices move around with purpose and transmit with purpose. We assert that
the same should be true in a simulation. However, modeling the purpose of actual people is not
possible. Therefore, the simulation gives the agents an alternative purpose: to stay alive. Each
agent has to ``eat'' in order to survive. Each turn, each agent may eat a unit of food if its
current location contains food. If it runs out of food, it dies. The goal of the agents is to stay
alive as long as possible. Thus, it must find sources of food within the simulation. This game-like
element provides the agents with goals and a purpose for communication.

As wireless networks become more widespread, the number of participants in a given network will
grow. Simulations that are not scalable may become too slow to simulate important test cases. For
this reason, Tecellate is designed to scale horizontally across many machines.

By modeling the communication at the physical layer and adding game-like elements to our simulation,
we hope to provide an interesting test bed for networking and AI algorithms. The game-like element
in our simulation will help ensure the agents move around purposefully in the simulation grid. The
physical layer simulation ensures that the communication problems which must be overcome for
reliable communication are similar to real world problems. The inherent scalability of the system
should allow simulation of arbitrarily large scenarios. While, we do not think our simulation
approach would be the final word in networks simulation, it should provide a useful platform for
experimentation.

\section{Literature Survey}

There are a few existing simulations of wireless ad hoc networks that operate above the link layer. One of these is ns2, which has been around for several years \cite{ns2}. ns2 simulates multiple kinds of networks and protocols, and it has been popular among researchers. However, it is difficult to scale above a few hundred nodes \cite{swans}. It has only been documented to scale to a few thousand nodes.

GloMoSim is a newer simulator \cite{glomosim}. It serves a similar purpose to ns2 and can simulate both wired and wireless networks, but it is more scalable in that it can run its event loop in parallel \cite{swans}. It can scale to tens of thousands of nodes.

SWANS is a simulator created purely for ad hoc wireless network research in response to the inadequacy of ns2 and GloMoSim for ad hoc wireless network research \cite{swans}. It has a different configuration/modeling style from other solutions which allows it to accurately simulate millions of nodes on 2004 hardware.

A specialized wireless ad hoc network simulator is $madhoc$, a system that attempts to accurately model wireless network characteristics in metropolitan settings \cite{madhoc}. The authors argue that other simulators' node movement algorithms do not accurately model real world environments, and consequently spend a great deal of effort modeling the characteristics of metropolitan environments.

These simulators differ from Tecellate in that they simulate a higher level of the network and provide no inherent goal for communicating agents. They also simulate radio signal propagation much more accurately, which affects the computation requirements drastically.


\section{Simulation Rules} \label{rules}

The simulation runs in discrete steps (\textbf{turns}). The conceptual ``length'' of a turn is
designed to simulate the time to send $k$ bytes, where $k$ is a configurable constant. Each turn, an agent may do any combination of the following: listen, broadcast, move, or die.

\subsection{Movement}

The simulation world consists of a $w$*$h$ grid of signed integers. If a cell's value $c$ is
non-negative, it is the terrain height. Otherwise, the cell is impassable. An agent can move at most
one cell up, down, left, or right per turn. However, after moving into a square, it must wait $c$
turns before it may make another move.

An agent's move may succeed or fail. If an agent tries to move into an impassible cell, it will
fail. It will also fail if it tries to move into a cell that was occupied by another agent at the
end of the last turn, \textbf{or} a cell that another agent attempts to move into in the same turn.
These cases are shown in figure \ref{mvt}.

\begin{figure*}[h!]
    \begin{center}
        \includegraphics[width=1in]{figures/mvt1.png}
        \includegraphics[width=1in]{figures/mvt2.png}
        \includegraphics[width=1in]{figures/mvt3.png}
    \end{center}
    \caption{Three main cases for movement.}
    \label{mvt}
\end{figure*}

It may seem more intuitive for both moves in case 3 to succeed, i.e. a move $Z\rightarrow Y$ by $A$
should succeed if agent $B$ moves $X\rightarrow Y$. However, allowing this case would introduce
large dependency graphs that would span a large geographical area. This kind of dependency graph
would greatly reduce the effectiveness of our horizontal scaling scheme based on geographical
partitioning described later in this paper.

\subsection{Messages}

Tecellate simulates radio communication. There are $N$ frequencies an agent can broadcast on or
listen too. An agent may broadcast a fixed number of bytes on a frequency and/or listen on a
frequency. If an agent listens to the same frequency it broadcasts to and no interference is
introduced, the agent will hear its own broadcast at the beginning of the next turn.

\begin{figure*}[h!]
    \begin{center}
        \includegraphics[width=6in]{figures/corrupt.png}
    \end{center}
    \caption{The probability of a byte being corrupted as a function of how far from the receiver
        the byte was broadcast.}
    \label{corrupt}
\end{figure*}

Like radio communication, the messages broadcast decay over distance. When an agent listens to a
frequency, the agent always receives data. However, if no one is broadcasting on that frequency, the
agent will receive random data. Since the message decays over distance, even if an agent hears it,
some parts of the message may be corrupted. The farther away a listener is from a sender, the more
corruption is in the message. As shown in Figure \ref{corrupt}, depending on the strength of the
broadcast, the message is less and less likely to be uncorrupted the farther the receiver is from
the transmitter.

\begin{figure*}[h!]
    \begin{center}
        \includegraphics[width=6in]{figures/combine.png}
    \end{center}
    \caption{The probability of a byte being combined into the received message as a function of
        how far from the receiver the byte was broadcast.}
    \label{combine}
\end{figure*}

Corruption of a message can also occur if more than one message is broadcast on the same frequency.
Figure \ref{combine} shows the probability that a byte will be heard by the receiver. If it is heard
it is logically anded with other bytes also heard. Thus, if there are two messages broadcast and the
receiver is close to one of the transmitters but far from the other there is a very low probability
of interference.

\subsection{Energy and Death}

An agent spends 1 unit of energy per turn. If it runs out of energy, it dies. It remains on the
field but cannot take any actions.

Energy gathering has not been implemented but will be an important feature in the future. It was
skipped for this stage of the project because it is simple to implement.

\subsection{Planned Features}

In general, agents are unable to perceive the world. In particular, agents cannot ``see'' terrain or
other agents. We are investigating ways to provide perceptions to agents in a way that match
artificial intelligence expectations well, but felt that the specifics were out of scope for this
project.

As mentioned earlier, energy cannot be gathered, only spent. Agents begin the simulation with fixed
and equal amounts of energy, which makes the energy feature useless in its current state.


\section{System Architecture}

The system can operate in one of two modes. In local mode, the simulation runs in one process with a
scenario defined in code. The remote mode is described below.

\subsection{Distributed Architecture}

\begin{figure*}
    \begin{center}
        \includegraphics[width=3in]{figures/arch0.pdf}
    \end{center}
    \caption{Basic architecture of the system.}
    \label{arch}
\end{figure*}

In order to support simulations on the scale of millions of agents, Tecellate can be run in a
distributed mode on many machines. Individual components communicate via TCP. The architecture of
this distributed version is shown in figure \ref{arch}.

\begin{figure*}
    \begin{center}
        \includegraphics[width=3in]{figures/arch1.png}
    \end{center}
    \caption{Grid-based method for distributing work across coordinators}
    \label{geoarch}
\end{figure*}

Agents are run in individual processes. In order to interact with the simulation, they connect to a
\textbf{coordinator}. Coordinators are responsible for evaluating discrete grid sections each turn
and communicating their results to each other when relevant. The organization of a square grid
section is shown in figure \ref{geoarch}.

The simulation starts with all agent and coordinator processes waiting and listening at individual
addresses. A master process reads a configuration file with their addresses and initial states,
sends information about how to connect to each other, waits for the connections to finish, and
instructs the coordinators to begin the simulation. It then exits and the other processes log
simulation data to configurable log files.

Each coordinator is responsible for a rectangular section of the grid. These sections are assigned
at the start of the simulation and do not change. Coordinators are connected with neighbor
coordinators that share adjacent grid sections. When an agent moves from one grid section to
another, the agent connects to the coordinator responsible for the new section.

\subsection{Coordinator Implementation Details}

There are two main types of threads associated with each coordinator. One kind of thread (A-thread)
requests agent data from neighbors, and then applies the simulation rules to the agents's previous
states based on their requested actions and the states of the agents at the neighbor coordinators.
There is only one of these threads. The other kind of thread (B-thread) serves requests for agent
information from neighbors, and there is one of these for each neighbor ($|N|$).

The A-thread and the B-threads communicate using two semaphores, \textbf{turnAvailable} and
\textbf{requestsServed}. The A-thread signals \textbf{turnAvailable} $|N|$ times when a turn has
been processed, and the B-threads each wait on it before serving a new request. When a B-thread has
served a request, it signals \textbf{requestsServed}. The A-thread waits on \textbf{requestsServed}
before it begins processing a new turn to avoid overwriting data that B-threads are still serving.
The system starts with \textbf{turnAvailable} unlocked for the B-threads.

This system allows processing and RPC serving to run in parallel, with processing running one step
ahead of RPC serving.

Processing a turn (in the A-thread) involves exchanging messages with each agent to get its moves
for the next turn. This may happen over TCP or in-memory communication channels depending on the
mode of operation.

\subsection{Issues and Possible Improvements}

The biggest issue with the current implementation is that distributed communication via TCP is
orders of magnitude slower than communication in memory. While this limiation is not surprising, it
is probably possible to improve communication speed by switching to a UDP-based protocol.

The application is highly multithreaded even within individual components. A modest run will have 40
or more threads. These are not OS-level threads but rather lightweight runtime-level threads.
Despite their lightweight nature, a large proportion are waiting for input most of the time, for
example to serve an RPC request. Our educated guess is that 60\% of threads are in this state at any
given moment. These threads may be wasting more CPU cycles than necessary when they are scheduled.

Performance improves when the language runtime is instructed to split the threads across multiple
CPU cores, but when this is done, a major memory leak manifests which makes the system unusable.
This memory leak may be caused by the language runtime, but we are still investigating the issue.

If agents cluster in one geographic area, one coordinator may have a significantly higher burden
than the others. This problem could be solved by allowing the master process to dynamically
reallocate the space assigned to each coordinator. This dynamic reassignment is very similar to the
problem of assigning many application instances to many servers, but in addition to simply trying to
distribute CPU and memory load evenly, we are also trying keep traffic between coordinators to a
minimum. It would take another paper to describe an effective algorithm to accomplish this goal, and
probably a simulator-simulator to test it.


\section{Validation}

iaweofiaw efoaw iefj



\section{Conclusion}

The rules outlined in this paper provide a reusable, simple platform for testing distributed artificial intelligence algorithms that are relevant to real-world applications. The system as described and implemented allow for a simulation of the game world to be run at a large scale, making use of new resources across multiple machines.


\nocite{*}
\bibliographystyle{acm}
\bibliography{bibliography}

\end{document}
