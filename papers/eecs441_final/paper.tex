\documentclass[12pt]{article}
\usepackage[margin=1in]{geometry}
\usepackage{graphicx}
\usepackage{setspace}
\usepackage{url}

\title{Tecellate: A Distributed Environment for Ad Hoc Wireless Network Simulations}
\author{
        Steve Johnson (srj15@case.edu)\\
        Case Western Reserve University\\
        \and Tim Henderson (tah35@case.edu)\\
        Case Western Reserve University
}
\date{\today}

\begin{document}
\doublespacing
\maketitle

% \begin{multicols}{2}

\begin{abstract}
    Tecellate is a simulation which autonomous agents can move in a grid and communicate using
    semi-realistic radio signals. The simulation itself can be distributed over multiple machines to
    allow for very large simulations. The system allows researchers to test various wireless network
    protocols and algorithms under arbitrarily harsh conditions.
    
     The usefulness of the simulation was tested using an approximated IP stack. Our initial results
    show that the simulation produces useful information, though more work is required to improve
    its accuracy.
\end{abstract}

\section{Introduction}

Network simulations have become important tools for researching mobile networks. Most research
groups do not have the ability to purchase and use thousands of mobile devices for research and have
thus turned to simulation. However, as many previous simulation authors have noted, simulations will
always sacrifice some element of realism. We have decided to take a novel approach to these
sacrifices, introducing a new way to test wireless mobile networks.

Tecellate simulates the physical layer of the network. All \textbf{agents} (actors in the
simulation) have a ``radio'' that sends and receives byte strings. These byte strings may be
corrupted, so agents must ensure that their link level connectivity is accurate. The simulation
includes complications like interference from other sources and exponential drop-off of power as
agents move away from the source of a transmission.

In the physical world, devices move around with purpose and transmit with purpose. We assert that
the same should be true in a simulation. However, modeling the purpose of actual people is not
feasible. Instead, we propose to give the agents an alternative purpose: to stay alive. Each agent
has to ``eat'' to survive. Each turn, it may eat a unit of food if its current location contains
food. If it runs out of energy, it dies. The goal of the agents is to stay alive as long as
possible. Thus, it must find sources of food within the simulation. This game-like element provides
the agents with goals and a purpose for communication.

As wireless networks become more widespread, the number of participants in a given network will
grow. Simulations that are not scalable may become too slow to simulate important test cases. For
this reason, Tecellate is designed to scale neatly across many machines.

By modeling the communication at the physical layer and adding game-like elements to our simulation,
we hope to provide an interesting test bed for networking and AI algorithms. The game-like element
in our simulation will help ensure the agents move around purposefully in the simulation grid. The
physical layer simulation ensures that the communication problems which must be overcome for
reliable communication are similar to real world problems. The inherent scalability of the system
should allow simulation of arbitrarily large scenarios. While, we do not think our simulation
approach would be the final word in networks simulation, it should provide a useful platform for
experimentation.

\section{Literature Survey}

There are a few existing simulations of wireless ad hoc networks that operate above the link layer. One of these is ns2, which has been around for several years \cite{ns2}. ns2 simulates multiple kinds of networks and protocols, and it has been popular among researchers. However, it is difficult to scale above a few hundred nodes \cite{swans}. It has only been documented to scale to a few thousand nodes.

GloMoSim is a newer simulator \cite{glomosim}. It serves a similar purpose to ns2 and can simulate both wired and wireless networks, but it is more scalable in that it can run its event loop in parallel \cite{swans}. It can scale to tens of thousands of nodes.

SWANS is a simulator created purely for ad hoc wireless network research in response to the inadequacy of ns2 and GloMoSim for ad hoc wireless network research \cite{swans}. It has a different configuration/modeling style from other solutions which allows it to accurately simulate millions of nodes on 2004 hardware.

A specialized wireless ad hoc network simulator is $madhoc$, a system that attempts to accurately model wireless network characteristics in metropolitan settings \cite{madhoc}. The authors argue that other simulators' node movement algorithms do not accurately model real world environments, and consequently spend a great deal of effort modeling the characteristics of metropolitan environments.

These simulators differ from Tecellate in that they simulate a higher level of the network and provide no inherent goal for communicating agents. They also simulate radio signal propagation much more accurately, which affects the computation requirements drastically.


\section{Simulation Rules} \label{rules}

The simulation runs in discrete steps (\textbf{turns}). The conceptual ``length'' of a turn is
designed to simulate the time to send $k$ bytes, where $k$ is a configurable constant. Each turn, an agent may do any combination of the following: listen, broadcast, move, or die.

\subsection{Movement}

The simulation world consists of a $w$*$h$ grid of signed integers. If a cell's value $c$ is
non-negative, it is the terrain height. Otherwise, the cell is impassable. An agent can move at most
one cell up, down, left, or right per turn. However, after moving into a square, it must wait $c$
turns before it may make another move.

An agent's move may succeed or fail. If an agent tries to move into an impassible cell, it will
fail. It will also fail if it tries to move into a cell that was occupied by another agent at the
end of the last turn, \textbf{or} a cell that another agent attempts to move into in the same turn.
These cases are shown in figure \ref{mvt}.

\begin{figure*}[h!]
    \begin{center}
        \includegraphics[width=1in]{figures/mvt1.png}
        \includegraphics[width=1in]{figures/mvt2.png}
        \includegraphics[width=1in]{figures/mvt3.png}
    \end{center}
    \caption{Three main cases for movement.}
    \label{mvt}
\end{figure*}

It may seem more intuitive for both moves in case 3 to succeed, i.e. a move $Z\rightarrow Y$ by $A$
should succeed if agent $B$ moves $X\rightarrow Y$. However, allowing this case would introduce
large dependency graphs that would span a large geographical area. This kind of dependency graph
would greatly reduce the effectiveness of our horizontal scaling scheme based on geographical
partitioning described later in this paper.

\subsection{Messages}

Tecellate simulates radio communication. There are $N$ frequencies an agent can broadcast on or
listen too. An agent may broadcast a fixed number of bytes on a frequency and/or listen on a
frequency. If an agent listens to the same frequency it broadcasts to and no interference is
introduced, the agent will hear its own broadcast at the beginning of the next turn.

\begin{figure*}[h!]
    \begin{center}
        \includegraphics[width=6in]{figures/corrupt.png}
    \end{center}
    \caption{The probability of a byte being corrupted as a function of how far from the receiver
        the byte was broadcast.}
    \label{corrupt}
\end{figure*}

Like radio communication, the messages broadcast decay over distance. When an agent listens to a
frequency, the agent always receives data. However, if no one is broadcasting on that frequency, the
agent will receive random data. Since the message decays over distance, even if an agent hears it,
some parts of the message may be corrupted. The farther away a listener is from a sender, the more
corruption is in the message. As shown in Figure \ref{corrupt}, depending on the strength of the
broadcast, the message is less and less likely to be uncorrupted the farther the receiver is from
the transmitter.

\begin{figure*}[h!]
    \begin{center}
        \includegraphics[width=6in]{figures/combine.png}
    \end{center}
    \caption{The probability of a byte being combined into the received message as a function of
        how far from the receiver the byte was broadcast.}
    \label{combine}
\end{figure*}

Corruption of a message can also occur if more than one message is broadcast on the same frequency.
Figure \ref{combine} shows the probability that a byte will be heard by the receiver. If it is heard
it is logically anded with other bytes also heard. Thus, if there are two messages broadcast and the
receiver is close to one of the transmitters but far from the other there is a very low probability
of interference.

\subsection{Energy and Death}

An agent spends 1 unit of energy per turn. If it runs out of energy, it dies. It remains on the
field but cannot take any actions.

Energy gathering has not been implemented but will be an important feature in the future. It was
skipped for this stage of the project because it is simple to implement.

\subsection{Planned Features}

In general, agents are unable to perceive the world. In particular, agents cannot ``see'' terrain or
other agents. We are investigating ways to provide perceptions to agents in a way that match
artificial intelligence expectations well, but felt that the specifics were out of scope for this
project.

As mentioned earlier, energy cannot be gathered, only spent. Agents begin the simulation with fixed
and equal amounts of energy, which makes the energy feature useless in its current state.


\section{Current State of the Project}

The simulation contains \textbf{agents} which must be sent stimuli to respond to. The current implementation consists of a \textbf{master process} which connects to several \textbf{coordinator processes} that are responsible for keeping the state of disjoint sets of agent processes.

The master process is responsible for connecting to the coordinators, sending them configurations, and waiting for their final responses. The coordinator connects to neighbor coordinators and agents, serves requests for information about the state of its agents in the last turn, and processes its own agents by requesting information from its neighbors and from its agents. In this way, the coordinators operate in lock-step, turn by turn, until all turns have been processed by all coordinators.

The current version of the simulation only supports agent movement and agent death. Agents are killed when they both occupy the same cell. The master process and coordinators do not attempt to balance load after the simulation has begun.


\section{Distributing Work}

Simulating thousands of agents and millions of grid cells simultaneously at a reasonable speed will require multiple machines. One process, the \emph{game master}, is responsible for starting the game and reporting the rules. $n_ac$ more processes, on the same machine or another, act as \emph{agent coordinators} responsible for launching and communicating with agent processes. They are responsible for delivering messages, taking and executing agent orders, and reporting success or failure. Coordinators communicate between themselves over persistent TCP connections.

Agents interact in two primary ways: messages and movement. Both are only effective within a certain proximity of the agent. These properties suggest that agents should be assigned to coordinators based on proximity, perhaps running as processes on the same server as their agent coordinator. There are multiple ways to perform this assignment. This paper considers three strategies: static assignment, fixed region distribution, and dynamic grid distribution.

\subsection{Static Assignment}

The simplest way to assign agents to coordinators is to assign $\frac{n_agents}{n_ac}$ agents to each agent coordinator at random or based on some positioning heuristic. No agents will ever change their coordinators.

This approach introduces significant overhead. Each agent coordinator will have to send all available information to all other coordinators every single turn because it cannot know which other coordinators contain agents close to its own.

\subsection{Fixed Region Distribution}

One way to alleviate the communication problem is to assign each agent coordinator a fixed portion of the grid to be responsible for. There will never be a recalculation of bounds and each coordinator has a constant set of neighbor coordinators to communicate with. Each turn, adjacent coordinators will exchange messages, and order confirmations for agents near their shared border, and full agent information transfers for agents moving from one coordinator's region to a neighbor's region.

This strategy requires the least communication between agent coordinators when agents move relatively little, but requires much communication when many agents are near region borders or move between borders often. It also introduces inefficiency in work distribution when disproportionate numbers of agents are gathered in small regions.

\subsection{Dynamic Grid Distribution}

Like fixed region distribution, dynamic grid distribution assigns each agent coordinator a region of the grid to be responsible for. However, the regions themselves are arranged in a meta-grid. This grid arragenement allows rows and columns to be resized while the simulation is running, partially alleviating the work distribution problem.

\subsection{Strategy for First Version}

The initial implementation of Tecellate will use static assignment to test communication protocols and basic mechanics. Then a fixed region distribution over a meta-grid will be introduced. If there is time, dynamic grid distribution will be implemented.

\section{Validation}

iaweofiaw efoaw iefj



\section{Conclusion}

The rules outlined in this paper provide a reusable, simple platform for testing distributed artificial intelligence algorithms that are relevant to real-world applications. The system as described and implemented allow for a simulation of the game world to be run at a large scale, making use of new resources across multiple machines.


\nocite{*}
\bibliographystyle{acm}
\bibliography{bibliography}

\end{document}
