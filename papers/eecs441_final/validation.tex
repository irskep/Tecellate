\section{Validation}

To validate the efficacy of our simulation the authors implemented a simplified network stack. The
physical layer is simulated by the simulation itself as described earlier. Since the physical layer
does not provide reliable communication between neighbors the link layer solves the problem of
making neighbor to neighbor communication reliable. Finally, the IP layer uses a simple active
routing protocol to deliver messages between any two connected agents in the network.

\subsection{The Link Layer}

The link layer provides reliable neighbor to neighbor communication as well as a ``HELLO'' protocol
for agents to announce their presence. Link layer packets have a simple structure. The first byte is
a command byte indicating what type of packet it is. The next four bytes are the physical address of
the sending agent (in this case the simulation assigned GUID). The following four bytes are the
physical address of the intended recipient. The body section is flexible in length and comes next.
The final 4 bytes are an error checking code.

To implement the error checking code the authors chose the 32 bit Cyclical Redundancy Check (CRC32)
using the IEEE polynomial. This is the same error checking code as used in the Ethernet protocol.
The CRC32 hash seems to be reliable enough for the purposes of this validation. 

\subsubsection{The HELLO Protocol}

In order for agents to find their neighbors an active introduction protocol was implemented. The
messages are transmitted at the link layer using a special link layer packet type (HELLO).
Periodically throughout the simulation an agent will transmit a HELLO packet to let its neighbors
know it is there. Each agent also listens for messages and records the HELLO's it has received in a
table together with a time out field. If it hears another message from an agent it has previously
heard from it refreshes the table with a new time out. When a neighbor's entry has timed out it is
removed from the neighbors table.

\subsection{The IP Layer}



