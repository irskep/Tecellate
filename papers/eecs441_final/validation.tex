\section{Validation}

To validate the efficacy of our simulation, the authors implemented a simplified network stack. The
physical layer is simulated by the simulation itself as described earlier. Since the physical layer
does not provide reliable communication between neighbors, the link layer solves the problem of
making neighbor to neighbor communication reliable. Finally, the IP layer uses a simple active
routing protocol to deliver messages between any two connected agents in the network.

\subsection{The Link Layer}

The link layer provides reliable neighbor to neighbor communication as well as a \texttt{HELLO}
protocol for agents to announce their presence. Link layer packets have a simple structure. The
first byte is a command byte indicating what type of packet it is. The next four bytes are the
physical address of the sending agent (in this case the simulation-assigned GUID). The following
four bytes are the physical address of the intended recipient. The body section is flexible in
length and comes next. The final 4 bytes are an error checking code.

To implement the error checking code the authors chose the 32 bit Cyclical Redundancy Check (CRC32)
using the IEEE polynomial. This is the same error checking code as used in the Ethernet protocol.
The CRC32 hash seems to be reliable enough for the purposes of this validation. 

\subsubsection{The \texttt{HELLO} Protocol}

In order for agents to find their neighbors an active introduction protocol was implemented. The
messages are transmitted at the link layer using a special link layer packet type (\texttt{HELLO}).
Periodically throughout the simulation an agent will transmit a \texttt{HELLO} packet to let its
neighbors know it is there. Each agent also listens for messages and records the \texttt{HELLO}s it
has received in a table together with a timeout field. If it hears another message from an agent it
has previously heard from, it refreshes the table with a new timeout. When a neighbor's entry has
timed out it is removed from the neighbors table.

\subsection{The IP Layer}

The IP layer implements a routing infrastructure to support agent to agent message exchange even
when the agents are not adjacent to one another. At the heart of this is an active \texttt{ROUTE}
announcement protocol and the distance vector routing algorithm. This is a fairly simplistic (and
naive) routing infrastructure. However, it demonstrates the efficacy of the simulation for
simulating ad hoc mobile networks.

The IP layer, like the link layer, implements a packet structure (termed hence forth as an IP
Datagram). The structure is similar to the link layer's packet and includes a redundant error
checking code. However, in addition to the to/from addresses it also includes a Time To Live (TTL)
field. There are only two message types at the IP layer messages and acknowledgements. Messages are
sent between any two agents in the network. Acknowledgements are sent only to direct neighbors to
confirm the receipt of the previous message sent. 

To build the routing tables each agent periodically broadcasts its current routing table. Initially,
the routing table only consists of the agents and it direct neighbors. However, as an agent hears
more routes over time, the agent adds the routes to its routing table based on the distance vector
protocol. Eventually, if the agents don't move, the distance vector algorithm is guaranteed to
converge on the optimal routing table. However, since the agents may move about without regard for
such concerns this approach to building routing tables is sub-optimal.

\subsection{Simulation Results}

To validate the simulator the authors ran several simulations using agents which communicated with
one another via the above described networking stack. The authors concluded two things from these
simulations. First, writing a network stack is difficult and ours performs poorly in the mobility
setting. Second, the simulator was effective in allowing the authors to test and debug the
networking stack.

\subsubsection{Poor Performance of the Example Network Stack}

The example network stack performs very poorly. The routing tables are very slow to converge even
with as few as 8 agents connected in a static line. From the experiments run it took anywhere from
500 to 1000 turns for the tables to converge. For comparison the neighbors table converged in 10 to
20 turns. Link layer messages conflicted which each other 17\% of the time, requiring the agents to
back off and resend their messages. This lead to even worse performance at the IP layer when sending
point to point messages between distant agents.

\subsubsection{Efficacy of Simulator}

The simulator was an effective and quick way to build a networking stack. The above stack was
written in less than 3 days and while it performs poorly, it works. The authors feel confident that
with more performance and usability (API) improvements the simulator will be a very friendly
environment for network programming. 

