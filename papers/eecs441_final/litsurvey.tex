\section{Literature Survey}

There are a few existing simulations of wireless ad hoc networks that operate above the link layer. One of these is ns2, which has been around for several years \cite{ns2}. ns2 simulates multiple kinds of networks and protocols, and it has been popular among researchers. However, it is difficult to scale above a few hundred nodes \cite{swans}. It has only been documented to scale to a few thousand nodes.

GloMoSim is a newer simulator \cite{glomosim}. It serves a similar purpose to ns2 and can simulate both wired and wireless networks, but it is more scalable in that it can run its event loop in parallel \cite{swans}. It can scale to tens of thousands of nodes.

SWANS is a simulator created purely for ad hoc wireless network research in response to the inadequacy of ns2 and GloMoSim for ad hoc wireless network research \cite{swans}. It has a different configuration/modeling style from other solutions which allows it to accurately simulate millions of nodes on 2004 hardware.

A specialized wireless ad hoc network simulator is $madhoc$, a system that attempts to accurately model wireless network characteristics in metropolitan settings \cite{madhoc}. The authors argue that other simulators' node movement algorithms do not accurately model real world environments, and consequently spend a great deal of effort modeling the characteristics of metropolitan environments.

These simulators differ from Tecellate in that they simulate a higher level of the network and provide no inherent goal for communicating agents. They also simulate radio signal propagation much more accurately, which affects the computation requirements drastically. They have sophisticated physics-based models of signal strength which must be updated and accounted for when sending and receiving transmissions, while we propose to use simpler probabilistic methods at the bit level.
